
\chapterimage{header.pdf} % Chapter heading image

\chapter{Reference Counting}

"Gerenciamento de memória? Pra que? \textit{Hardware} é tão barato, não vou desalocar a memória". "Sou vida loka, faço pipoca de panela aberta e não libero memória".
"Só libera a memória quando quer impressionar a estagiária nova?". Se você pensa dessa maneira, parabéns campeão, você é um dos meus, 
está liberado para ir assistir \textit{The Big Bang Theory}. Mas para aqueles que são caretas e gostam de viver de forma correta vamos apresentar uma outra solução.
O gerenciamento de memória é uma das maiores dores de cabeça dos programadores \textit{C}, fazer essa atividade manualmente com 
\textit{malloc()} e \textit{free()} funciona, porém se você esquecer um \textit{free()} poderá causar um \textit{memory leak}, se você usar o \textit{free()} de 
forma errada, você pode corromper seu programa. Essa atividade não precisa ser tão árdua se utilizarmos algumas regras durante o gerenciamento da memória.


\section{Como funciona?}
\textit{Reference couting} é uma técnica de gerenciamento de memória bem simples, seu funcionamento é baseado em um contador interno 
do objeto. Esse contador se inicia em 1 (um), quando o objeto é criado. Quando o programador precisar utilizar o objeto ele chama o método
\textit{Ref}, assim o contador interno do objeto é incrementado em 1, quando este não for mais utilizado, o método \textit{UnRef}
deve ser chamado, o contador interno será decrementado em 1. Quando esse contador interno chegar em zero, quer dizer que ele não 
é mais referenciado e pode ser liberado. Vamos seguir com um pequeno passo-a-passo.

\vspace{5 mm}

\begin{itemize}
  \item person\_new   | contador = 1  /* Criando o Objeto, contador inicia em 1 */
  \item person\_ref   | contador = 2  /* Programador retendo o Objeto, incrementa o contador */
  \item person\_unref | contador = 1  /* Programador liberando o Objeto, decrementa o contador */
  \item person\_unref | contador = 0  /* Objeto não é mais necessário, decrementa o contador */
  \item Memória será liberada;        /* Objeto não é mais referenciado por ninguém e será liberado */
\end{itemize}
\vspace{5 mm}
Reparem que chamamos duas vezes o método \textit{person\_unref}, um \textit{unref} é para o método \textit{person\_new} e outro porque utilizamos o método \textit{person\_ref} para reter
o objeto.

\section{Exemplo: Objeto Person}

Para exemplificarmos a técnica de \textit{reference couting}, iremos dar o exemplo de um objeto \textit{Person} que contém \textbf{nome}, \textbf{sobrenome} e \textbf{idade}. 
Esse exemplo é apenas demonstrativo, portanto não se apeguem a nomes e tratamento de erros, a idéia principal é passar o conceito 
de \textit{reference counting}. Para esse nosso exemplo iremos implementar os métodos:
\vspace{5 mm}
\begin{itemize}
  \item person\_new: Responsável por alocar o objeto;
  \item \_person\_destroy: É um método estáfico responsável por liberar o objeto (memória);
  \item person\_ref: Método responsável por reter o objeto (incrementa o contador);
  \item person\_unref: Método responsável por liberar o objeto;
  \item person\_print: Método para imprimir as informações de Person;
\end{itemize}

\section{Implementando o header}

\textit{Header}? O que é isso? Não é só criar um \textit{main.c} e colocar tudo lá dentro? Isso mesmo campeão, quanto orgulho de ti, mas como nossos leitores são 
frescos e gostam de coisas mais organizadas iremos criar o arquivo \textit{person.h}, colocar as assinaturas dos métodos e até fazer 
uns comentários para ninguém sair falando que nosso código não é documentado. Estou me sentindo um programador \textit{java}, vamos 
fazer um diagrama de estados??? NOOOOTTT.

\belowcaptionskip=-10pt
\begin{lstlisting}[label=some-code,caption=person.h]
#ifndef _PERSON_H_
#define _PERSON_H_

typedef struct {
    char *first_name;
    char *last_name;
    unsigned int age;
    unsigned ref;
} Person;

/*  Method to create the persong object */
Person* person_new(char *first_name,
                   char *last_name,
                   unsigned int age);

/* Retain the object */
void person_ref(Person *obj);

/* Release the object */
void person_unref(Person *obj);

/* Print object information */
void person_print(Person *obj);


#endif /* _PERSON_H_ */
\end{lstlisting}

\section{Implementando o método new}

Chegou o momento de implementarmos o método que irá criar o nosso objeto, ou seja, o método responsável por alocar a memória. TESÃOOO, já sinto um friozinho na barriga, OPS!
Terminarei de escrever essa parte no banheiro e farei muita FORÇA para sair TUUUDO cerrtoo. Meu caro leitor, antes de continuar a parte técnica, vou me abrir um pouco com 
vocês, e também acender um fósforo. Vamos falar um pouco de merda, mas é só isso que tem nesse documento, ok ok, tu me entendeu. Existe prazer maior no mundo que dar uma 
c4g4d4 e programar em \textit{C} ao mesmo tempo? Se existe eu desconheço. Eu particularmente acho estranho as pessoas não falarem muito que programam durante o seu momento no
trono, porra é muito bom. Na real c4g4r e programar tem muito em comum, em ambos os casos quando você termina, enche a boca e fala orgulhoso: "Esse é meu filhão!" Voltando para a
merda do código reparem que \textit{obj->ref} foi inicializada com 1.

\belowcaptionskip=-10pt
\begin{lstlisting}[label=some-code,caption=person\_new()]
Person *person_new (char *first_name, 
                    char *last_name, 
                    unsigned int age)
{
        if (NULL == first_name) {
            printf("Invalid first_name!\n");
            return NULL;
        }

        if (NULL == last_name) {
            printf("Invalid last_name!\n");
            return NULL;
        }

        Person *obj = (Person *)malloc(sizeof(Person));
        if (NULL == obj) {
                printf("%s Out of Memory!\n", __FUNCTION__);
                return NULL;
        }

        obj->first_name = strdup(first_name);
        obj->last_name = strdup(last_name);
        obj->age = age;
        obj->ref = 1;   /*  Reference Counting */

        printf("Creating object[%p] Person\n", obj);
        return obj;
}

\end{lstlisting}

\vspace{50mm}

\section{Método person\_ref}
Agora vamos implementar o método que irá reter o nosso objeto. Lembre-se que para cada Ref no objeto um UnRef deve ser chamado, caso contrário um leak ocorrerá.

\belowcaptionskip=-10pt
\begin{lstlisting}[label=some-code,caption=person\_ref()]
void person_ref(Person *obj)
{
    if (NULL == obj) {
       printf("Person Obj is NULL");
       return;
    }
    obj->ref++; /*  Incrementando nosso contador */
}
\end{lstlisting}

\section{Método UnRef}

Lembram para que serve o método UnRef? Pra porra nenhuma não é a resposta correta. Ele é responsável por decrementar o contador interno do objeto e verifica 
se o objeto ainda é referenciado, caso não seja mais, um método para liberar a memória será chamado.

\belowcaptionskip=-10pt
\begin{lstlisting}[label=some-code,caption=person\_unref()]
void person_unref(Person *obj)
{
    if (NULL == obj) {
       printf("Person Obj is NULL");
       return;
    }

    /*  Decrementa o contador e 
     *  verifica se esta igual a 0 
     */
    if (--obj->ref == 0) {
        printf("Memory Release obj[%p]\n", obj);
        _person_destroy(obj);
    }
}
\end{lstlisting}

\section{Liberando a memória}

Quando implementamos o método \textit{person\_new()} alocamos algumas informações na memória, e esse é o momento de honrarmos o que temos no meio das pernas, 
assumirmos nossas responsabilidades e mandarmos as informações para caso do caralho, desculpe pessoal pelo ataque de raiva, estou sem café. 
Como eu estava dizendo, esse é o momento de liberarmos a memória e vivermos felizes sem \textit{memory leak}. 
\vspace{50mm}
\belowcaptionskip=-10pt
\begin{lstlisting}[label=some-code,caption=\_person\_destroy()]
static void _person_destroy(Person *obj)
{
    if (NULL == obj) {
        printf("Person object is NULL!\n");
        return;
    }

    if (NULL != obj->first_name) {
        free(obj->first_name);
    }

    if (NULL != obj->last_name) {
        free(obj->last_name);
    }

    free(obj);
}
\end{lstlisting}

\section{Implementando o main.c}

Dentro do \textit{main.c} que iremos utilizar os métodos que implementamos acima. Reparem que para cada método \textit{person\_new()} tem um \textit{person\_unref()} associado. 
E você está livre para fazer testes, tente comentar a linha \textit{person\_unref(father);} e repare que a memória não será liberada.

\belowcaptionskip=-10pt
\begin{lstlisting}[label=some-code,caption=main.c]
#include <stdio.h>

#include "person.h"

int main()
{
    Person *father = person_new("Jose", "Rico", 65);
    Person *mother = person_new("Beth", "Perigueti", 21);

    person_print(father);
    person_print(mother);

    person_ref(father);
    person_unref(father);

    /* New method - Unref */
    person_unref(father);
    person_unref(mother);

    return 0;
}
\end{lstlisting}


\section{Criando o Makefile}

Como eu sou um cara muito legal e to ligado que a grande maioria dos leitores são preguiçosos, irei fazer um \textit{Makefile}, se eu fosse um cara sacana eu faria um \textit{autohell}, 
hehehe.

\belowcaptionskip=-10pt
\begin{lstlisting}[label=some-code,caption=Makefile]
CC      := gcc

CFLAGS  := -W -Wall -Werror -I.

BIN     := person

SRC := main.c person.c
OBJ := $(patsubst %.c,%.o,$(SRC))

%.o: %.c
                $(CC) $(CFLAGS) -o $@ -c $<

all: $(OBJ)
                $(CC) $(CFLAGS) -o $(BIN) $(OBJ)

clean:
                $(RM) $(BIN) $(OBJ) *.o $(LIB)
\end{lstlisting}

\section{Conclusão}

Podemos tirar alguma conclusão boa desse capítulo? O cara que escreveu esse capítulo é um anão, grosso, mal educado e burro. Ok, isso é verdade, mas vamos pular os detalhes. 
Acredito que o gerenciamento de memória em \textit{C} não precisa ser uma tarefa tão dolorosa, se seguirmos algumas regras básicas podemos tornar essa atividade mais fácil. 
Esse exemplo é simples, mas a idéia é a mesma para problemas complexos. Lembrando que o código completo está no \textit{github}, na pasta \textit{code}, dentro do capítulo 
\textit{reference.}


